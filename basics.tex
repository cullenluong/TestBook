\documentclass{book}
\usepackage[backend=bibtex,style=verbose-trad2]{biblatex}
\bibliography{refbasics}
\usepackage{hyperref}
\usepackage{amssymb}
\usepackage[draft]{todonotes}   % notes showed
\hypersetup{
    colorlinks=true,
    linkcolor=blue,
    filecolor=magenta,      
    urlcolor=cyan,
}


\usepackage{listings}
\begin{document}
\frontmatter
\title{Points of Views of the World}
\author{Cullen Luong}
\maketitle



\let\oldemptyset\emptyset
\let\emptyset\varnothing


% Introductory chapters
\chapter{Preface}


Raw Infinity and God.

Infinity in all contexts.


computations and functions
Base level abstractions ( logic, category theory, set theory)

The core consists of the most basic axioms and rules.

A universe consists of a subset of possible statements, with additional possible axioms/assumptions.

Theories concern themselves with the multiverse.


The subjects with recursiveness built in are learning,language,math (category theory/set theory,math,physics), and psychology (emotions and mind).

To achieve an intuiton as strong as a the pohysical world. 


The first person in history that we know by name wasn't some king, or a hero, but rather an accountant in ancient Sumeria  \autocite[1]{Sumeria:1}. 

Language is unique to humanity. The ability to fully describe the world, and ourselves is is not found in any other animals. Sure animals can and do exhibit intelligent behaviors from time to time. 

University of and nor, if then. 

We invented numbers and writing and art to keep track of the phases of the moon, of whether or not my neighbor Sam owes me some copper for that time he threw up in my house. 

The story of human abstraction is the record of ideas in its purest form. What do I mean by ideas? I mean not just the possibilities that the world throws at us, but anything that is even possible for our minds to imagine.  ibelieve that are capable of knowning and imagining anything possible. I am not a Christian, but the idea of humanity being in the image of God, what does that mean exactly? What is the image of God? God is omnipotent, omniscient, omnipresent. God is totality and the universe, t . It is the ability tocreate something out of nothing. 


The first thing people think about when you ask them what math is. The stereotype is of course that mathematicians sole job is to calculate number all day long. 
"Hey so and so can you calculate the amount tip?" , "No I'm not very good with numbers".
"Okay so if that's not what mathematicians and physicists do what do they do? Witchcraft?"

Of course that isn't what mathematicians and physicists deal with. They deal with how and why, and what? But its often hard to specifically describe what mathematics is to the average layperson. 


I want to describe a consistent way of thinking and dealing with the world of abstraction and rational thinking. ]]

I will not be discussing an important question - 
what is the difference between representations/symbols and the real thing?

How are words defined?
The world of symbols vs the 'real' world.
A set of axioms and rules vs the model.
Peano axioms vs natural numbers?


One can say that math/logic is absolutely anthing that deals with abstraction and rational thinking. 
There are countless  areas of mathematics from algebra,calculus, to geometry and topology. THe number of possible topics is for intents and purposes infinite. 


First lets imagine the universe of all possible permutations,combinations,lists of symbols. This is everything.
And kind of useless, because if you can do any statement is valid, then there's noting interesting to say. Every statement is as valid as any other.
This is hy we introduce axioms and definitions. We can then categorize and divide the universe into statements of vary9ing validity. Some statements are invalid, others valid.  





A huge part of the basics of math is simply recording what we see and observe in the physical world. This might seem like an overly physicists viewpoint, but I hope to convince you of its truth.

For example

 
 The picture we see is what the world would be if it was nice and simple. The next pictue, the next frame will be simple. But the more the frames, the longer the time, the more the world will deviate. 
 
Empiricism vs Rationalism 
 
 


If you are wrong then you weren't aware of the possibilities, you're premise is wrong, you're steps are wrong, or all the above...

'Genius' and creativity.It requires a little daring. To not be afraid of making a mistake, of doing the wrong thing. \\
 I want a more exploratory kind of book, but yet have a certain formalism to it. Many books start with theory then application. Yet this is not how much of math is invented and discovered. 

 The point of formalism is to be rigorous. So rigorous that you can have a computer verify line by line every step of the proof. 

The best way to learn a concept is simply to repeat it, again and again in as many different way as you can into your head. 

When reading, simply read. Then take notes of your own afterwards. 

Terminology, basic operations, basic questions, examples,
0 basic problems that tests basic concepts separately
1 intermediate problems that involve more than one concept 
2 counter-intuitive type problems
3 challenges
paradoxes and counter-intuitive problems/results.
I want you to be able to visualize results whenever possible. 

Rise of axiomatic systems. You start with a small certain amount of statements that are taken to be absolutely true. Using inference rules and logic you then create new statements.
But forcing yourself to think about it that way can very hard.
Think about it this way. YOu can make statements that says you can construct a element, say a point or line. You then have something that can relate this point to other points (even to itself!). 






People often forgot previous concepts. They can apply it in a specific way, but people
don't know how to generalize well. 
Remember previous concepts. Everything builds on one another. 

Stop trying brute force memorize 
Look at computational, mathematical, engineering/physical, common sense points of views.

So I want to develop the concept of numbers. I also want to develop the intuition of geometry and the world we live in.
Numbers + abstraction = everything. 

%% start with thesis
Numbers are simply ordered symbols.They are representative of something
Second perspective the concept of numbers is something unto itself. 
Okay what isn't a number? Symbols can be anything you want. You just have to be able to differentiate one from another. 

The world isn't totally random. Every picture is related to one another. The fact that there can be any kind order means. You could say math is recording the world.

Things are different one another. An obvious fact. 
You can know and understand something from something simpler. 
Abstraction is a small self contained language. 



once you have a set elements and operations, all that is left is every combination and permutation of them.
The very concept of symbols means we get the concept of the three laws
of logic for free. symbols/things and the manipulation/change of them are the most basic.

Type theory is a framework for doing logic. It is a system of inference rules and patterns of reasoning.
Set theory is built upon logic. It is defined by membership and the properties that define the membership. It focuses on the objects themselves.
Category theory is built upon objects and mapping/function compositions. It focuses on the change itself.

Humans most powerful tools are to record observations and preserve knowledge through oral and written means.
To systematically classify and order.


first the raw board
abstraction board
infinitely more abstractions boards

greatest abstraction is to reduce entire fields and concepts to a single letter
least abstractions to write everything out.
The key is somewhere in between usually.

mechanics/under the hood -> abstraction and reversed?

basic and important theorems and operations
save all little details to own pdf

creating new variables and dividing new ones.
master basics and consider basic examples first

When considering a(n open) problem consider the axioms and assumptions and see what fields can help solve it.

What math and programming does is provide a universal language for exploring everything there is and can be.

set of statements/operations

\chapter{History}
We started math with the concept of numbers. 
We record the passing of the days, the moons, the seasons, and the years. 

Of course maths purview has expanded to include almost eerything.Math has 
But what math doesn't explain, the question of what is it really?




My goal is to force you to imagine the world state of the abstraction. 

models -> axioms
axioms -> model(s)

1. Awareness of All Objects and Possibilities
2. Grouping and interpreting  this universe



\chapter{Imagining all Possibilities, 0th Abstraction}
All possible combinations.

For each new length, write out all symbols and new sentences.



Manipulation of objects.

Objects can be written symbols/letters/words.
They can real objets, sounds ,sgihts,touch/feel.
They can be memories,desires,emotions,concepts.
They can be anything. 

Awareness and memory of objects.
Will to manipulate/add/remove/ \texit{change} said objects. 


All possible statements,constructions,symbols.
brute force will always eventually work.
Try everything

Discrete , limits of discrete (continuous,functional,ect)
So given some list of symbols that represent anything you want, imagine every possibility up to n elements, of every symbol.


The -> and = symbols. 

Addition of units.
Subtracting units.
Replacing units. 

adding to a structure.
deleting part of a structure.
modifying structure

Linking between groups of units.

When writing a program. YOu branch out and link between the possible universes. 


\chapter{Logic and Category Theory}
What is truth?
Truth is only defined a context.
The study of truth is epistemology. 
You cannot know anything for certain. 


\section{Category Theory}
The basic lesson we learn from Category Theory is the concept of functions, and not in the analyatcial sense, but in a more extremely general sense.

You put in any object then you get an object back. This object might be the same, it might be completely different.


Levels of abstraction, creating new units that related to the old.
To practically make sense of the infinity we should go for a more refined approach then simply brute forcing everything. To achieve this we make use of some form of abstraction.

In practice what scientist and mathematicians do is logic and programming. 

Now category theory and set theory might be frameworks where you speak of fundamental objects around. 

You write out statements, subroutines, scripts. They do something, and you are often concerned with validity and understandibilty. 


So what is an abstraction? An abstraction is a simplification that reduces systems to a more managable units. 
For exampleooooool

Metaclasses 
Numbers,Images,Colors, Ect.

Class/Types -> Integers, Matrix,Reals,ect

Represenation

Universe of all possible statements%
statements are precise descriptions.  
creating a new statement 

Log is about the validity of statements. 

introduce programming and mathematical notation from the start

Current modern math is axiomatic. 

	Type systems are more modern, and probably more suited for programming purposes. However set theory is still extremely useful and any mathematical textbooks will constantly use set notations and ideas.
	
	Set theory is built on logic.
	Logic tells us how/whether each new statement is valid given some preceding statements using inference rules.
	Types assigns labels to pieces of codes and gives an indicator of how it will behave. This seems very vague, but it also makes it more powerful than a set.

Universal Concepts/take aways of membership function/indicator
asking if something if variable is of type whatever.
W
object,operators,relations

lets assume that everything can described by or approximated by discrete objects (A practial computtional viewpoint)


A good text to read would be BBj \cite{DUMMY:1}

I. Axioms // 
There may be more than on models that satisfy the axioms. 
If a statement cannot be proved under the axioms there are two possibilities. 

A logial system can described by how powerful it is.  
\url{https://math.stackexchange.com/questions/1713602/about-zfc-peanos-axioms-first-order-logic-and-completeness}

1. The statement is true in some models and false in some other models \\ 
2. The statement has a truth statement but cannot be proved under the axioms.  \\

II. Operators  and Objects //


Think of the universe of elements and operations. 


classify each symbol as either an object,operation,or relation


State which stores symbols = tape for turing machines, register for register machines ,ect

Program which tells us how to manipulate these symbols

The concept of moving on to  different state and recursive thinking is essential to all abstractions.


Constructive vs axiomatic.Axioms are restrictions. There can be multiple models for an axiom. 



\chapter{The Universe}
Imagine every possible state of every possible length (super permutations and De-brujin Sequences). 
Image every possible graph( evewry psossible edges) of every possible size.

When we image what these symbols mean attach meanings between these graphs, and that leads us to totality.

\section{Tally Marks}
all possible tally marks

all possible graphs of these tally marks

Ask yourself what these tally marks mean and how to say it.

How do you go from these tally marks to language?

That is the essence of turing machines, to set theory and cateogry theory, then to logic.

\section{Number Systems}

.
\chapter{Philospophy and Knowledge}

\section{Epistemology}
epistemology 
Is knowledge justified true belief?
gettier problem
https://en.wikipedia.org/wiki/List_of_paradoxes#Self-reference
Conclusions: The universe is inherently unknoweable to a certain extent(even without taking into account quantum mechanics and what not). 

\section{Mind in a Vat}



\section{Godel's Completeness and Incompleteness Theorems}

No axiomatic system that is at least as powerful as peano's arthmitic can be simultaneously,
sound, complete, effective

A system is complete if every true statement can be proved within that system. 

\chapter{Computational Models}


All machines consistst of states and instruction(program),



\section{Turing Machines}
The system consist of tapes with symbols ( at least 0,1 and null ) written on it.

At any given point the program is on a particular point in the tape with particular internal symbol signifying its program.



\section{Register Machine}
Turing Tape
Register Machine
2-state 3-symbol Turing machine
Lambda Calculus
Game of Life

The intuition is this.

Infinite Rules for manipulating an infinite amount of symbols over space and time.  



\chapter{Algorithm Considerations}
How to arrive at state?
How simple is program?
How fast is program?
How accurate is program?


\section{Epistemology}
What is knowledge?
How do you what you know to be true?

Is it justified true fact?
Counter Example:
(from paper) You deduce a man is hiding behind a a certain rock, because

In other words right for the wrong reason.

Wrong with (incomplete) right reason.


\section{Problems}
Busy Beaver
Halting Problem
Oracle Problem
Justified True Fact is not knowledge
Godel's Incompleteness


\section{Set Operations}
Product of sets5

\section{Class and Type Systems}

truth value functions and logic control
\section{Iteration,Maps, and Recursion}
for loops, while loops are often used for traversals or
accumulation of values


\section{Universality of Logic Gates}
/url{https://www.quora.com/How-do-you-prove-mathematically-that-NAND-gates-can-compute-any-logical-function}
Use series of ors for each truth ouput and series of ands for each false outputs

\section{Database Functions}
sql and pandas
querying,indexing,loc

Merge, join, and concatenate

\section{Common Datastructures}
dictionaries/sets
(singly/doubly) linked lists
queues and stacks
vectors and arrays

binary trees
nary trees
graph

priority queue, balanced binary tree,heaps
a set is solely defined by its members that is a membership function
subsets, powersets,
object oriented programming

\url{https://bartoszmilewski.com/2014/10/28/category-theory-for-programmers-the-preface/}


\url{https://bartoszmilewski.com/2014/10/28/category-theory-for-programmers-the-preface/}
Observe that you can do anything (computable) through function composition. What this means is when you input something, you of course get something out. This can always be accomplished by  a sequence of applying functions, ie, function compositions.

Note that in this universe we remember only one thing, and that is our previous input.

A category is defined by objects and arrows.

There are two definitions of functions - first in the programming sense which is completely general and secondly in the mathematical sense which only applies to numbers.



\url{https://bartoszmilewski.com/2014/10/28/category-theory-for-programmers-the-preface/}




\chapter{Proof and Problem Techniques}
logic inference rules
proof by contradiction
modus whatever

induction

Transforming/reducing or approximating problems to something that you already know. 
For example linearity a nonlinear equation.



\section{Higher Order Logic}






\section{Pigeonhole Principle}
\section{Proof By Contradiction}
\section{Contraceptive}
\section{Induction}
Proof that P(0) is true.
Assume that P(n) is true. (Weak)
Assume that P(n) is true for all numbers from 0 to n. (Strong)
Prove that P(n+1) is true. 
Thus P(n) is true for all n = 0 to infinity. 

Exercise. Prove that weak induction is a subset of strong induction.

Assume P(n) is provable by weak induction. 


\section{Proof By Cases and Elimination/Enumeration of Possibilities}

\chapter{Category Theory}
\chapter{Functors and Monads}

\chapter{Paradoxes}
\section{References}
rudin, Terrence tao analysis
going from the finite to the infinite requires taking limits
lets also go from the finite to the infinite

rearranging terms in an infinite sum
adding zeros and subtracting zeros in an infinite sum

\chapter{Group Theory and Permutations}
All groups are isomoprhic to some symetry group. In other words
group operations,permutations and symmetry all describing the same concept. 

\chapter{Metamath Theory Stuff}
associativity and commutativity of operations is often quite important. 
distribution is also useful

In physics projection-like operators are extremely useful. 




\chapter{Machine code and assembly}
\backmatter


\chapter{References}
Stanford Encyclopedia of Philosophy
wikitionary
mathworld,physicsworld, lubos motl, terry tao blog
wikipedia
stackexchange
quanta
rudin,
book of proofs
napkin is good

bbj computability

art of computer programming

algorithms

b ok org
libgen.io



\url{https://en.wikipedia.org/wiki/Foundations_of_geometry#A_critique_of_Euclid}


Blablabla said Nobody  \autocite{DUMMY:1}.


\newpage

\printbibliography


\end{document}


